%%==================================================
%% chapter2.tex for BIT Master Thesis
%% version: 0.1
%% last update: Nov 8th, 2017
%%==================================================
\chapter{自适应控制和半参数模型}
\label{chap:2 }
\section{问题描述}
存在无法预测的不确定性,系统的输出就会产生偏差,即便是采用离散时间角度设计控制律。反馈从一开始就是为了解决这种偏差和不确定性而引入的,自适应控制从本质山说是一种非线性反馈\upcite{Guo2003}。不过当系统存在不确定性时,是否一定存在一种自适应控制律能够使得系统闭环稳定?这是在设计自适应控制之前首先要回答的问题。然而直到20世纪90年代末和本世纪初,郭雷院士等学者\upcite{XieGuo1999,ZhangGuo2002}创造性地提出反馈机制的最大能力和局限这一重要命题,同时针对类似下面表达式\eqref{eq:Guo}给出的一些基本控制系统进行定量探索,才引起来广泛的关注。
\begin{equation}%
\label{eq:Guo}
y_{k+1} = \theta f(y_{k}) + u_{k} + \omega_{k+1}
\end{equation}
其中,$u_{k}$,$y_{k}$,$\omega_{k}$ 分别是系统在$k$时刻的输入、输出和干扰,而$\theta$和$f(\cdot)$分别是待估计的参数和未知函数。系统\eqref{eq:Guo}的不确定性主要表现为函数$f(\cdot)$的未知性,这常常被称为非参数不确定性(nonparametric uncertainties);而$f(\cdot)$的绝对值随自变量$y_{k}$绝对值的增长速率决定了反馈机制能力的上限\upcite{Guo2002},在随后的研究中针对高阶情形中多元函数$f(\cdot)$,这一条件被扩展到Lipschitz条件下的规律。可以这样理解,如果$f(\cdot)$增长的比较快,那么就不存在使得系统\eqref{eq:Guo}镇定的自适应反馈控制律。分析出反馈机制的上下并给出数学证明,确实不是一件容易的事情。这一理论很直观地说明了系统的不确定性的大小常常表现为系统未知函数或者参数的变化剧烈程度,这些结论从控制稳定性和可行性的角度对系统的不确定性给出了一定的定量描述。

目前关于反馈机制及能力的研究主要针对离散时间控制系统,这与离散时间控制器的广泛应用这一大趋势不谋而合,并且一开始就考虑到了非参数不确定性,因而具有较好的理论高度和普适性。反馈机制及能力给出了对不确定性系统的认识,从理论上看,不是所有的未知系统都可以设计出稳定的自适应控制律;用但是同时也留下了一个问题:如果系统属于自适应反馈控制可稳定的范围内,那么如何设计出合适的自适应反馈控制律使系统达到满意的性能?一般来说,不同的被控对象自然结构和参数都不同,但是否存在一种较为通用的模型可以刻画大部分面对的被控对象?这样的问题可以类比经典控制理论,毕竟线性系统和微分方程这些通用工具在几十年的控制理论中起到了关键的作用。

虽然数字化控制器输出到被控对象的信号是离散时间,但是实际的过程是连续时间变化的。一般的离散时间非线性控制往往忽略了这一点,就导致这些研究侧重于理论和离散时间仿真分析。在系统\eqref{eq:NARX}中,把被控对象当成完全非线性模型,忽略了他们的线性特性,也就是说很多被控对象虽然具有较大的不确定性,但是在一定程度上是可以当作线性对象处理,只不过同时具有非线性部分,导致实际的线性部分参数与理论值有较大的偏差;对于不确定性系统,也就常常表现为同时具有参数不确定性(parametric uncertainties)和非参数不确定性。例如电机驱动系统简化处理就是一个二阶线性模型,一般的PID控制也可以获得还算不错的效果;只是在负载未知或者变化较大,以及存在一些非线性特性如库伦摩擦等情况下,会影响PID的控制效果。这一事实说明了实际系统表现出的线性特性有助于解决非线性自适应控制问题,不可忽视。这样看来,一般的非线性系统都可以看作是由线性参数部分和非线性的非参数部分组成。