%%==================================================
%% conclusion.tex for BIT Master Thesis
%% modified by yang yating
%% version: 0.1
%% last update: Dec 25th, 2016
%%==================================================

\begin{conclusion}

系统存在不确定性,一直是控制理论研究和工程应用的难点,特别是在离散时间控制中。本文针对一般同时具有参数和非参数不确定性的离散时间被控对象,在已有半参数理论的基础上,研究并完善半参数建模与分析、估计与控制方法,充分利用了系统的先验信息和输入输出历史数据,并考虑了离散时间控制的实时性和精度。主要工作总结如下:
\begin{enumerate}
\item 对反馈控制、非线性控制和自适应控制的发展历程、主要趋势和技术特点进行了系统的归纳和总结,指出最小二乘等线性估计方法的不足,研究了人工神经网络在非线性建模与控制中的应用与难点,介绍了一种在前馈神经网络的实时性应用中具有很大潜力的超限学习机方法。
\item从回顾反馈机制能力极限等前沿问题出发,针对系统具有参数不确定性和非参数不确定性的特点进行了分析,将系统分为参数描述的线性部分和非参数描述的非线性部分,概括出了一种半参数建模方法,并对实际系统常用的先验信息进行了数学描述。
\item 研究了一种充分先验信息的、基于集合的自适应估计方法,即信息浓缩估计,用于解决半参数系统的参数不确定性;设计了二维情形下的具体实现算法和步骤,并在数值仿真实验中测试和验证实际效果;归纳总结了信息浓缩估计算法的优缺点,提出了一些计算复杂度的优化方向。
\item 介绍了半参数自适应估计与控制问题的一般描述,围绕解决神经网络中经典迭代学习算法的实时性等问题,阐述了超限学习机及其变体的算法特点与非线性建模思路;将超限学习机引入到半参数系统的非参数部分估计中,设计了基于超限学习机的半参数自适应控制算法,并通过仿真和实验进行了验证。
\item 在机器人与伺服电机等运动体的轨迹跟踪控制中存在大量的不确定性,将基于信息浓缩估计和超限学习机的半参数建模、估计与控制方法应用到机器人场景中伺服电机控制问题中,对运动控制问题进行了定制化设计;在仿真实验中进行算法验证,结果表明,半参数自适应控制大大提高了轨迹跟踪精度。
\end{enumerate}

基于超限学习机的半参数自适应控制算法的特点主要体现在,充分利用原有系统的先验信息和输入输出历史数据,结合信息浓缩估计和超限学习机等高效且新颖的算法,其建模与控制思路在解决具有较大不确定性的被控对象时具有很强的一般性。但是,由于控制中利用的估计算法本身的缺点,还有一些问题需要进一步研究与完善:
\begin{enumerate}
\item 信息浓缩估计算法的完善。本文目前设计了二维情形下的信息浓缩估计算法,并在自适应运动控制中进行了验证。二维情形可以在一定程度上解决一般运动控制场景如伺服电机的跟踪控制问题,但是对于涉及到高阶模型,含有多个参数的情况如机器人末端的力反馈控制等问题,可能需要进一步设计高维情形中的信息浓缩估计算法,主要思路不变。
\item 多变量系统的自适应控制实现。本文主要涉及到输入和输出是单一变量的情形,多输入多输入情形的建模与分析思路与本文大致相同,只是涉及到更多变量的信息浓缩估计与神经网路逼近,可能需要考虑使用本文第三章论述的计算复杂度优化策略。
\end{enumerate}
%end of conclusion
\end{conclusion}