%%==================================================
%% chapter4.tex for BIT Master Thesis
%% version: 0.1
%% last update: Nov 8th, 2017
%%==================================================
\chapter{离散时间半参数自适应控制}\label{chap:4}
自适应控制中的的辨识是为了控制问题,上一章设计了二维情形下的信息浓缩估计算法去辨识线性部分的未知参数。本章将在第三章的基础上设计非参数部分的估计算法,主要借助于一种机器学习算法,然后根据两部分的估计结果去设计自适应控制律在去解决随动控制问题。
\section{问题描述}\label{sect:4.1}
再次考虑第二章给出的半参数模型
\begin{equation}%
\label{eq:4.semi-u}
y_{k+1} = \bm{\theta}^{T}\bm{\phi_{k}}+u_{k}+f(\bm{\psi_{k}})+\omega_{k+1}
\end{equation}
其中,$\bm{\theta}$是未知参数向量,$f(\cdot)$是未知函数。同理,令
\begin{equation}
z_{k} = f(\psi_{k}) + \omega_{k+1}
\end{equation}

针对系统\eqref{eq:4.semi-u},自适应估计与控制问题的一般表述是:给定关于$\bm{\theta}$,$f(\cdot)$和$\omega_{k+1}$的先验信息,如何根据实时产生的一系列输入输出数据$\{y_{k},u_{k};k=1,2,\ldots\}$,去估计未知参数$\bm{\theta}$和$z_{k}$?然后根据$\bm{\theta}$和$z_{k}$的估计(预测)值去设计合适的自适应控制输入$u_{k}$使得$k+1$时刻的输出$y_{k+1}$能跟踪上期望的输出$y_{k+1}^{*}$。

第三章已经给出了$\bm{\theta}$的估计算法,因此剩下的部分可以分为两个核心问题,一方面是设计$f(\cdot)$的估计算法,另一方面是设计控制输入$u_{k}$的表达式。假设第二章设计出的未知参数的估计值为$\hat{\bm{\theta}}_{k}$,下面需要设计出$\breve{z}_{k}$的表达式。因此,根据必然等价原理,设计出的控制输入的一般表达式为:
\begin{equation}\label{eq:4.uk}
u_{k}^{*}=y_{k+1}^{*}-\bm{\phi_{k}}^{T}\cdot\hat{\bm{\theta}}_{k}-\breve{z}_{k}
\end{equation}

非线性部分的辨识与控制器的设计是本章重点解决的问题。

\section{非参数部分的估计}\label{sect:4.2}

\subsection{最近邻估计}
前面的章节用信息浓缩估计解决了参数不确定性的辨识问题,半参数自适应控制在提出来的时候,针对非参数部分使用的是最近邻估计。最近邻估计主要针对的是一维参数情形($d_{1}=1$且$d_{2}=1$),即下面的一维半参数系统
\begin{equation}%
\label{eq:4.1ord}
y_{k+1}=\theta y_{k}+u_{k}+f(y_{k})+\omega_{k+1}
\end{equation}
这里,标量$\theta$是未知的,为参数不确定性部分,同样$f(\cdot)$是非参数不确定性部分,满足Lipschitz条件。最近邻估计不显示地估计$f(\cdot)$部分,而是直接估计
\begin{equation}\label{eq:4.nne.g}
\eta_{k} = \theta y_{k}+f(y_{k})
\end{equation}
这个整体。这样,在用信息浓缩算法估计出参数$\theta$之后,移除这个估计值$\hat{\theta}$,就可以获得非参数部分的估计。

在系统\eqref{eq:4.1ord}中,由于Lipschitz条件的存在,不管是非参数部分还是参数部分都主要和输出值$y_{k}$相关,这样输出值$y_{k}$相近的时刻,那么$\eta_{k}$的值也应该相差不大。因此最近邻估计就主要利用到了这一点,其主要思想在于寻找和当前时刻输出值$y_{k}$最相近的时刻,然后进行比较计算。

首先定义最近邻时刻为
\begin{equation}\label{eq:4.it}
i_k=\mathop{\arg \min}\limits_{i<k}|y_k-y_i|
\end{equation}
$i_k$是历史输出值中和当前时刻最接近的时刻。那么有如下等式成立
\begin{equation*}
\begin{array}{lll}
&\eta_k&=\eta_k-z_{i_k}+z_{i_k}\\
&&=[\theta y_k+f(y_k)]-[\theta y_{i_k}+f(y_{i_k})+w_{i_k+1}]+z_{i_k}\\
&&=[\theta(y_k-y_{i_k})+z_{i_k}]+[f(y_k)-f(y_{i_k})-w_{i_k+1}]
\end{array}
\end{equation*}
一般来说,上面等式中的最后一行中$f(y_k)-f(y_{i_k})$的值接近零,并且干扰项比较小。因此,可以取整体表达式\eqref{eq:4.nne.g}的估计为
\begin{equation}\label{eq.4:g.est}
\begin{array}{lll}
\hat{g}_k&\triangleq\hat{\theta}_k(y_t-y_{i_k})+z_{i_k}
&=\hat{\theta_k}(y_k-y_{i_k})+(y_{i_k+1}-u_{i_k})
\end{array}
\end{equation}

接着,令
\begin{equation}\label{eq.4:bk}
\begin{array}{lll}
\bar{b}_k\triangleq \max\limits_{i\leq k}y_i
         =\max(\bar{b}_{k-1}, y_k)\\
\underline{b}_k\triangleq \min\limits_{i\leq k}{y_i}
                =\min(\underline{b}_{k-1}, y_k).
\end{array} 
\end{equation}

最后基于设计出下面的控制律
\begin{equation}\label{eq:4.uk}
u_{t}=\left\{
\begin{array}{cc}
  -\hat \eta_k+y_{k+1}^* & \text{if } |y_k-y_{i_k}|\le D \\
  -\hat \eta_k+\frac12(\bar b_k+\underline b_k) & \text{if } |y_k-y_{i_k}|> D \\
\end{array}
\right.
\end{equation}
这里,$D$是一个合适的常数值。

由此看出,最近邻估计可以比较准确地给出一维情形下非参数部分的估计,从而解决半参数系统的自适应控制问题。但是有一定的局限性,主要体现在:
\begin{enumerate}
\item 由于最近邻时刻$i_k$的计算需要遍历所有的历史值,因此,随着时间的增长,理论上需要无限的内存用来存储历史数据。这给实施带来了困难。
\item 由于在计算$\hat{g}_k$时没有对干扰值作特别处理,因此适用于干扰值的有界范围比较小且变化比较慢的系统。
\item 没有充分利用历史数据和关联先验信息,只是利用最近邻时刻的数据,计算结果不一定是最优的。
\end{enumerate}

以上的缺点存在,导致现有的半参数自适应控制存在很大的局限性。本章借助于其他的非线性辨识算法去设计新的非参数部分估计算法。

\subsection{神经网络}
机器学习解决的问题从数学上主要分为分类和回归。其中分类处理的数据主要是离散型数据,回归解决的对象和思路跟控制中的估计、预测十分类似。而神经网络作为一种十分有效的回归与预测算法,在非线性辨识中应用十分广泛。神经网络主要由输入层、隐含层和输出层组成(最简单的神经网络可能没有隐含层),依据不同的激活函数或者不同的学习算法,种类十分丰富。本章主要讨论一种十分常见的单隐层前馈神经网络(Single hidden layer feedforwar network, SLFN),即具有一个输入层、一个隐含层以及一个输出层的前馈型神经网络。

一个具有$N_{I}$个输入神经元、$N_{O}$个输出神经元以及$N_{H}$个隐含层神经元的单隐层前馈神经网络的激活函数为
$g(\cdot)$。假设存在任意$N_{s}$个不同的输入输出样本对
$$\{\bm{x}_{i},t_{i}\},\ i=1,2,\ldots,N_{s}$$
这里输入向量为
$$\bm{x}_{i}=[x_{i1},x_{i2},\ldots,x_{in}]\in \mathcal{R}^{N_{I}}$$
输出向量为
$$\bm{t}_{i}=[t_{i1},t_{i2},\ldots,t_{in}]\in \mathcal{R}^{N_{O}}$$
则神经网络的数学模型输出为
\begin{equation}%
\label{eq:4.slfn}
\begin{split}%
o_{j}&=\sum_{i=1}^{N_{H}}\beta_{i}g(w_{i},b_{i},\bm{x}_{i})
&=\sum_{i=1}^{N_{H}}\beta_{i}h_{i}(\bm{x}_{i})
&=h(\bm{x}_{i})\beta
\end{split}
\end{equation}

人工智能的发展一直推动着神经网络的研究,特别是2012年之后,深度学习(Deep Learning, DL)在图像分类、语音识别、自然语言处理等应用领域中获得了巨大成功\upcite{ZhengChenZhang2014}。目前大部分的深度学习算法大部分是基于神经网络建立。对于深度学习来说,隐含层的数量级为十几层到几百层不等,甚至上千层。深层网络结构和特征学习思想是深度学习的两大主要特点。与深度学习相比,超限学习机在不丢失精度的条件下具有快速的优势,比较见表\ref{tab:elm-dl}所示。

\begin{table}
\centering
\caption{超限学习机与常见深度学习算法的性能比较}\label{tab:elm-dl}
\begin{tabular*}{0.9\textwidth}{@{\extracolsep{\fill}}cccc}
\toprule
算法名称	&测试精度($\%$)	&训练时间 \\
\midrule
H-ELM	&99.14	&281.37s\\
Multi-layer ELM	&$99.03\pm0.04$	&281.37s\\
Deep Belief Networks(DBN)	&98.87	&20580s(5.7 hours)\\
Deep Boltzmann Machine(DBM)	&99.05	&68246s(19 hours)\\
Stacked Auto Encoders(SAE)	&98.6	&>17 hours\\
Stacked Denoising AutoEncodes(SDAE)	&98.72	&>17 hours\\
\bottomrule
\end{tabular*}
\end{table}

\subsection{ELM及其变体}

\section{自适应控制器设计}
半参数自适应控制

\section{仿真实例}

\section{本章总结}