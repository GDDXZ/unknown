%%==================================================
%% abstract.tex for BIT Master Thesis
%% version: 0.1
%% last update: Nov 8th, 2017
%%==================================================

\begin{abstract}
传统方法对解决同时存在参数和非参数不确定性的系统存在设计、解释和实现等局限性,特别是在现代数字化离散时间控制系统中。半参数自适应控制给出了一种新的理论框架,解决两种不确定性同时出现的系统。而现有的半参数自适应估计与控制还存在准确性、实时性等问题,需要进一步完善。本文系统归纳和总结了半参数系统的建模与分析,以及估计与控制算法设计思路,将计算几何、超限学习机、数据驱动等方法引入到控制器设计中,并针对机器人、电机控制等实际问题的解决进行了初步尝试。

首先,研究了一般系统中广泛存在的不确定性特点,对自校正调节器、模型参考自适应、鲁棒控制、无模型自适应等经典控制方法进行了对比分析;在阐述反馈机制能力极限问题的基础上,介绍了一种半参数建模与分析方法,突出了参数不确定性和非参数不确定性的各自特征;研究系统先验信息的数学描述和信息浓缩估计算法,用于解决系统参数部分不确定性的估计。

然后,描述了二维参数情形下信息浓缩估计算法的主要步骤和基本问题,对信息浓缩变换中直线和多边形的几何关系进行了详细地分析,同时设计了信息浓缩变换的算法流程图;在实际的被控对象中仿真前面设计的信息浓缩估计算法,讨论了信息浓缩估计的计算复杂度优化问题,总结了信息浓缩估计的主要优缺点。

接着,给出了半参数自适应轨迹跟踪问题的数学描述,并分析了半参数模型的自适应估计和控制问题;分析了超限学习机的算法特点,并利用合适的超限学习机变体算法设计了非参数部分的估计算法;在信息浓缩和超限学习机的自适应估计基础上,设计了针对半参数系统轨迹跟踪问题的自适应控制器,用仿真实例对比测试和验证控制算法的性能。

最后,分析了机器人轨迹跟踪和伺服电机运动控制中多种不确定性存在等控制难点,将基于超限学习机的半参数自适应估计与控制应用到运动控制中,解决机器人运动中电机的控制问题,通过仿真对比实验测试和验证半参数自适应运动控制的性能。在文章的最后归纳总结了本文的主要工作,并指出了本课题今后的继续研究方向。

\keywords{超限学习机;信息浓缩估计;半参数系统;自适应控制}
\end{abstract}

\begin{englishabstract}

The traditional method has some limitations such as the design, interpretation and implementation of the system for solving the simultaneous and non-parametric uncertainties, especially in the modern digital discrete-time control system. Semi-parametric adaptive control gives a new theoretical framework to solve the system of two kinds of uncertainties simultaneously. However, the existing semi-parametric adaptive estimation and control still have some problems, such as accuracy and real-time, which need to be further improved. This thesis systematically summarizes the modeling and analysis of semiparametric systems, as well as the design of estimation and control algorithms, and introduces the methods of computational geometry, extreme learning machine and data-driven into the design of controllers. In addition, this series of proposed methods has been applied in the robotic manupulator and motor control as initial attempt for pratical problems. 

First of all, this thesis studies the characteristics of uncertainties existing in the general system and comparatively analyzes the classical control methods such as self-tuning regulator, model reference adaptive, robust control, model-free adaptive and so on. A novel kind of method called semi-parametric modeling and analysis method is introduced, highlighting the respective characteristics of parametric uncertainty and non-parametric uncertainty. The mathematical description and information enrichment estimation algorithm of the priori information of the system are used to solve the problem of system estimation of the uncertainty of the parameter part.

Second, the main steps and basic problems of the information concentration estimation algorithm in the case of two-dimensional parameters are described. The geometric relationships between lines and polygons in the information transformation are analyzed in detail. At the same time, the flow chart of the algorithm of information condensation transformation is designed. The simulation examples of the proposed information concentration estimation algorithm are depicted. The computational complexity optimization of information concentration estimation has been discussed. The main advantages and disadvantages of the information concentration estimation have been summarized.

Third, the mathematical description of semi-parametric adaptive trajectory tracking is given, and the problem of adaptive estimation and control of semi-parametric models is analyzed. The algorithm characteristics of the extreme learning machine are analyzed. The non-parametric part estimation method based on the appropriate variant algorithm of extreme learning machine has been proposed. According the self-adaptive estimation of information concentration and extreme learning machine, one kind of  adaptive controller is designed for the trajectory tracking problem of semi-parametric systems. The simulation examples are used to compare the performance of testing and verifying control algorithms.

Finally, the control difficulties of robot trajectory tracking and servo motor motion control with multiple uncertainties are analyzed. Semi-parametric self-adaptive estimation and control based on extreme learning machine are applied to solve the motor control in robot motion problem. The performance of semi-parametric adaptive motion control have been tested and verified through simulation comparison experiment. At the end, the main work of this thesis are summarized and the future research direction of this topic are discussed.
   
\englishkeywords{extreme learning machine; information concentration estimation; semi-parametric system; adaptive control}

\end{englishabstract}
