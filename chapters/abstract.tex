%%==================================================
%% abstract.tex for BIT Master Thesis
%% version: 0.1
%% last update: Nov 8th, 2017
%%==================================================

\begin{abstract}
传统方法对控制同时具有参数和非参数两种不确定性的系统,在设计、解释和实现上存在较大的局限性,特别是在现代数字化离散时间系统的应用中。半参数自适应控制给出了一种新的理论框架,适用于两种不确定性同时出现的系统。而现有的半参数自适应估计与控制还存在准确性和实时性都不佳的问题,需要进一步完善。本文系统地论述了半参数建模、估计与控制算法的设计思路,将计算几何、超限学习机、数据驱动等方法引入到控制器设计中,并应用到机器人与电机控制中,对解决实际问题进行了初步尝试。

首先,研究了一般系统中广泛存在的不确定性特点,对自校正调节器、模型参考自适应、鲁棒控制、无模型自适应等经典控制方法进行了对比分析;在阐述反馈机制能力极限问题的基础上,介绍了一种半参数建模与分析方法,突出了参数不确定性和非参数不确定性的各自特征;对常用的先验信息进行了数学描述,从而引出信息浓缩估计算法,用于解决系统参数部分不确定性的估计。

其次,描述了二维参数情形下信息浓缩估计算法的基本问题和主要步骤,对信息浓缩变换中直线和多边形的几何关系进行了详细分析,同时设计了信息浓缩变换的算法流程图;在实际的被控对象中仿真前面设计的信息浓缩估计算法,讨论了信息浓缩估计的计算复杂度优化问题,总结了信息浓缩估计的主要优缺点。

然后,给出了半参数自适应轨迹跟踪问题的数学描述,并分析了半参数模型的自适应估计和控制问题;分析了超限学习机的算法特点,并利用合适的超限学习机变体算法设计了非参数部分的估计算法;在信息浓缩和超限学习机的自适应估计基础上,设计了针对半参数系统轨迹跟踪问题的自适应控制器,用对比方法在仿真实例中测试和验证控制算法的性能。

最后,分析了机器人轨迹跟踪和伺服电机运动控制中多种不确定性同时存在的控制难点,将基于超限学习机的半参数自适应理论与方法应用到运动控制场景中,解决机器人运动中电机的控制问题,通过对比仿真实验,测试和验证半参数自适应运动控制的性能。文章的最后归纳总结了本文的主要工作,并指出了本课题的继续研究方向。

\keywords{超限学习机;信息浓缩估计;半参数自适应;运动控制}
\end{abstract}

\begin{englishabstract}

The traditional methods have some limitations such as designing, interpretation and implementation of control  systems for simultaneously solving parametric and non-parametric uncertainties, especially in modern digital discrete-time control systems. Semi-parametric adaptive control gives a novel theoretical framework to solve the plants with these two kinds of uncertainties. However, the existing semi-parametric adaptive estimation and control still have some problems, such as accuracy and real-time properties, which need to be further improved. This thesis systematically discusses the designing methods of semi-parametric modeling, estimation and control algorithms, and brings the concepts of computational geometry, extreme learning machine and data-driven into the design of controllers. In addition, this series of methods proposed in this thesis have been applied in robotic manipulator and motor control as useful attempts for practical problems. 

First of all, this thesis studies the characteristics of uncertainties existing in the general system and comparatively analyzes the classical control methods such as self-tuning regulator, model reference adaptive, robust control, model-free adaptive and so on. A novel kind of method called semi-parametric modeling and analytical method is introduced, which highlights the respective characteristics of parametric uncertainty and non-parametric uncertainty. The mathematical descriptions of priori information are established for typical systems so that information concentration estimation algorithm is used to solve the estimation problem of parametric uncertainty part.

Second, the main steps and basic problems of the information concentration estimation algorithm in the case of two-dimensional parameters are described. The geometric relationships between lines and polygons in the information concentration process are analyzed in detail. At the same time, the algorithm flow chart of information concentration transformation is designed. The simulation examples of the proposed information concentration estimation algorithm are depicted. The computational complexity optimization of information concentration estimation has been discussed. What's more, the main advantages and disadvantages of information concentration estimation have been summarized.

Third, the mathematical descriptions of semi-parametric adaptive trajectory tracking are given, and the problem of adaptive estimation and control of semi-parametric model is analyzed. The algorithm characteristics of the extreme learning machine are analyzed. The non-parametric part estimation method based on the appropriate variant algorithm of extreme learning machine has been proposed. According the adaptive estimation of information concentration and extreme learning machine, one kind of adaptive controller is designed for the trajectory tracking problem of semi-parametric systems. The simulation examples are used to test and verify the performance of the proposed control algorithms.

Finally, the control difficulties of robot trajectory tracking and servo motor motion control with multiple uncertainties are analyzed. Semi-parametric adaptive estimation and control based on extreme learning machine are applied to solve the motor control in robot motion problem. The performance of proposed adaptive motion control has been tested and verified through simulation comparison experiments. In the end, the main contents of this thesis are summarized and the future research directions of this topic are discussed.
   
\englishkeywords{extreme learning machine; information concentration estimation; semi-parametric adaptive; motion control}

\end{englishabstract}
