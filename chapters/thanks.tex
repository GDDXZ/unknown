%%==================================================
%% thanks.tex for BIT Master Thesis
%% modified by yang yating
%% version: 0.1
%% last update: Dec 25th, 2016
%%==================================================

\begin{thanks}

本论文的工作是在导师***教授的指导下独立完成的。两年半的时光匆匆划过,虽有苦有泪,但它始终是我这一生中一段重要的旅程。研究生阶段让我在更好的完善自己的同时,也丰富了我的知识,开阔了眼界。而如今,我能够顺利的完成我的毕业论文,也是因为一直有那些人陪伴在我的左右,在我错误时指导我,在我失意时陪伴我,在我受挫时鼓励我的那些人。在此论文完成之际,我要对曾经帮助过我的师长、亲人、同学、朋友表示诚挚的谢意。

感谢敬爱的导师***教授在我的学习和生活中给与的帮助和教诲。在入学之初,***老师就开始悉心安排我硕士期间的课题方向,给了我一些比较有意思小问题,既有理论上的,也有工程上的,让我首先尝试去解决,给提供了一些基础性的书籍和资料。这些问题的解决使我逐渐进入机器人和自适应控制的研究当中。后来在深入开展课题的研究中,***老师经常与我不断交流。每次遇到一些障碍,***老师总能让我醍醐灌顶,找到一些新的思路。***老师深厚的理论素养、忘我的工作热情和优秀的的人格魅力,以及一直以来对学生无微不至的关怀,让我终身受益。

感谢***教授、***教授在学习和科研上的指导,两位学识渊博、平易近人的老师使我获益匪浅;感谢***老师给我提供和补充其他相关领域的知识和方法;感谢***和***师兄,为我提供了许多仿真实验上的有益帮助和一些细节上的耐心指点;感谢***等实验室的同学与我一起探讨学术、工程上的难题。

感谢***、***、***和***等同学在研究生期间给我的帮助,今后必将怀念与你们一起在自动化学院6号楼一起奋斗的日子,以及那些美好的时光。

在此,我也要特别感谢我的父母及其他亲人朋友,你们的支持和帮助,一直是我前进的动力,能够顺利完成硕士学位,也离不开你们在经济上的支持,感恩你们无私的爱!

最后,感谢本论文中所引用的参考文献的作者,你们的文章给了我无限的启迪和智慧,感谢本论文的评审老师和答辩委员会老师,同时也感谢**学院的所有任课老师们,感谢各位老师的不惜吝教。

\end{thanks}
